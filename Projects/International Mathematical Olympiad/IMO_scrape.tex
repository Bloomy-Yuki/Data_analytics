% Options for packages loaded elsewhere
\PassOptionsToPackage{unicode}{hyperref}
\PassOptionsToPackage{hyphens}{url}
%
\documentclass[
]{article}
\usepackage{amsmath,amssymb}
\usepackage{iftex}
\ifPDFTeX
  \usepackage[T1]{fontenc}
  \usepackage[utf8]{inputenc}
  \usepackage{textcomp} % provide euro and other symbols
\else % if luatex or xetex
  \usepackage{unicode-math} % this also loads fontspec
  \defaultfontfeatures{Scale=MatchLowercase}
  \defaultfontfeatures[\rmfamily]{Ligatures=TeX,Scale=1}
\fi
\usepackage{lmodern}
\ifPDFTeX\else
  % xetex/luatex font selection
\fi
% Use upquote if available, for straight quotes in verbatim environments
\IfFileExists{upquote.sty}{\usepackage{upquote}}{}
\IfFileExists{microtype.sty}{% use microtype if available
  \usepackage[]{microtype}
  \UseMicrotypeSet[protrusion]{basicmath} % disable protrusion for tt fonts
}{}
\makeatletter
\@ifundefined{KOMAClassName}{% if non-KOMA class
  \IfFileExists{parskip.sty}{%
    \usepackage{parskip}
  }{% else
    \setlength{\parindent}{0pt}
    \setlength{\parskip}{6pt plus 2pt minus 1pt}}
}{% if KOMA class
  \KOMAoptions{parskip=half}}
\makeatother
\usepackage{xcolor}
\usepackage[margin=1in]{geometry}
\usepackage{color}
\usepackage{fancyvrb}
\newcommand{\VerbBar}{|}
\newcommand{\VERB}{\Verb[commandchars=\\\{\}]}
\DefineVerbatimEnvironment{Highlighting}{Verbatim}{commandchars=\\\{\}}
% Add ',fontsize=\small' for more characters per line
\usepackage{framed}
\definecolor{shadecolor}{RGB}{248,248,248}
\newenvironment{Shaded}{\begin{snugshade}}{\end{snugshade}}
\newcommand{\AlertTok}[1]{\textcolor[rgb]{0.94,0.16,0.16}{#1}}
\newcommand{\AnnotationTok}[1]{\textcolor[rgb]{0.56,0.35,0.01}{\textbf{\textit{#1}}}}
\newcommand{\AttributeTok}[1]{\textcolor[rgb]{0.13,0.29,0.53}{#1}}
\newcommand{\BaseNTok}[1]{\textcolor[rgb]{0.00,0.00,0.81}{#1}}
\newcommand{\BuiltInTok}[1]{#1}
\newcommand{\CharTok}[1]{\textcolor[rgb]{0.31,0.60,0.02}{#1}}
\newcommand{\CommentTok}[1]{\textcolor[rgb]{0.56,0.35,0.01}{\textit{#1}}}
\newcommand{\CommentVarTok}[1]{\textcolor[rgb]{0.56,0.35,0.01}{\textbf{\textit{#1}}}}
\newcommand{\ConstantTok}[1]{\textcolor[rgb]{0.56,0.35,0.01}{#1}}
\newcommand{\ControlFlowTok}[1]{\textcolor[rgb]{0.13,0.29,0.53}{\textbf{#1}}}
\newcommand{\DataTypeTok}[1]{\textcolor[rgb]{0.13,0.29,0.53}{#1}}
\newcommand{\DecValTok}[1]{\textcolor[rgb]{0.00,0.00,0.81}{#1}}
\newcommand{\DocumentationTok}[1]{\textcolor[rgb]{0.56,0.35,0.01}{\textbf{\textit{#1}}}}
\newcommand{\ErrorTok}[1]{\textcolor[rgb]{0.64,0.00,0.00}{\textbf{#1}}}
\newcommand{\ExtensionTok}[1]{#1}
\newcommand{\FloatTok}[1]{\textcolor[rgb]{0.00,0.00,0.81}{#1}}
\newcommand{\FunctionTok}[1]{\textcolor[rgb]{0.13,0.29,0.53}{\textbf{#1}}}
\newcommand{\ImportTok}[1]{#1}
\newcommand{\InformationTok}[1]{\textcolor[rgb]{0.56,0.35,0.01}{\textbf{\textit{#1}}}}
\newcommand{\KeywordTok}[1]{\textcolor[rgb]{0.13,0.29,0.53}{\textbf{#1}}}
\newcommand{\NormalTok}[1]{#1}
\newcommand{\OperatorTok}[1]{\textcolor[rgb]{0.81,0.36,0.00}{\textbf{#1}}}
\newcommand{\OtherTok}[1]{\textcolor[rgb]{0.56,0.35,0.01}{#1}}
\newcommand{\PreprocessorTok}[1]{\textcolor[rgb]{0.56,0.35,0.01}{\textit{#1}}}
\newcommand{\RegionMarkerTok}[1]{#1}
\newcommand{\SpecialCharTok}[1]{\textcolor[rgb]{0.81,0.36,0.00}{\textbf{#1}}}
\newcommand{\SpecialStringTok}[1]{\textcolor[rgb]{0.31,0.60,0.02}{#1}}
\newcommand{\StringTok}[1]{\textcolor[rgb]{0.31,0.60,0.02}{#1}}
\newcommand{\VariableTok}[1]{\textcolor[rgb]{0.00,0.00,0.00}{#1}}
\newcommand{\VerbatimStringTok}[1]{\textcolor[rgb]{0.31,0.60,0.02}{#1}}
\newcommand{\WarningTok}[1]{\textcolor[rgb]{0.56,0.35,0.01}{\textbf{\textit{#1}}}}
\usepackage{graphicx}
\makeatletter
\def\maxwidth{\ifdim\Gin@nat@width>\linewidth\linewidth\else\Gin@nat@width\fi}
\def\maxheight{\ifdim\Gin@nat@height>\textheight\textheight\else\Gin@nat@height\fi}
\makeatother
% Scale images if necessary, so that they will not overflow the page
% margins by default, and it is still possible to overwrite the defaults
% using explicit options in \includegraphics[width, height, ...]{}
\setkeys{Gin}{width=\maxwidth,height=\maxheight,keepaspectratio}
% Set default figure placement to htbp
\makeatletter
\def\fps@figure{htbp}
\makeatother
\setlength{\emergencystretch}{3em} % prevent overfull lines
\providecommand{\tightlist}{%
  \setlength{\itemsep}{0pt}\setlength{\parskip}{0pt}}
\setcounter{secnumdepth}{-\maxdimen} % remove section numbering
\ifLuaTeX
  \usepackage{selnolig}  % disable illegal ligatures
\fi
\IfFileExists{bookmark.sty}{\usepackage{bookmark}}{\usepackage{hyperref}}
\IfFileExists{xurl.sty}{\usepackage{xurl}}{} % add URL line breaks if available
\urlstyle{same}
\hypersetup{
  pdftitle={IMO\_Data},
  pdfauthor={sophie},
  hidelinks,
  pdfcreator={LaTeX via pandoc}}

\title{IMO\_Data}
\author{sophie}
\date{2024-04-01}

\begin{document}
\maketitle

\hypertarget{an-r-markdown-file-to-scrape-the-data-from-imo-official-website}{%
\section{\texorpdfstring{An R Markdown file to scrape the data from
\href{https://www.imo-official.org/}{IMO official
website}}{An R Markdown file to scrape the data from IMO official website}}\label{an-r-markdown-file-to-scrape-the-data-from-imo-official-website}}

\hypertarget{general-idea}{%
\subsection{General Idea}\label{general-idea}}

fist we need to get the link to the problems and use some variable date
to scrape the tables and change it to a data frame, then clean each
table individually and lastly parsing them all together and saving the
result as a .csv file for later cleaning and analysis.

\begin{Shaded}
\begin{Highlighting}[]
\CommentTok{\#Loading the tidyverse}
\FunctionTok{library}\NormalTok{(rvest)}
\FunctionTok{library}\NormalTok{(tidyverse)}
\end{Highlighting}
\end{Shaded}

\begin{verbatim}
## -- Attaching core tidyverse packages ------------------------ tidyverse 2.0.0 --
## v dplyr     1.1.4     v readr     2.1.5
## v forcats   1.0.0     v stringr   1.5.1
## v ggplot2   3.5.0     v tibble    3.2.1
## v lubridate 1.9.3     v tidyr     1.3.1
## v purrr     1.0.2     
## -- Conflicts ------------------------------------------ tidyverse_conflicts() --
## x dplyr::filter()         masks stats::filter()
## x readr::guess_encoding() masks rvest::guess_encoding()
## x dplyr::lag()            masks stats::lag()
## i Use the conflicted package (<http://conflicted.r-lib.org/>) to force all conflicts to become errors
\end{verbatim}

\hypertarget{scraping-the-first-table}{%
\subsection{Scraping the first table}\label{scraping-the-first-table}}

We need to establish a fist table, as a basis for all our clean ones to
get parsed into. and as such we start by using the link and some
functions to do so

\begin{Shaded}
\begin{Highlighting}[]
\CommentTok{\#declaring a url variable}
\NormalTok{url }\OtherTok{\textless{}{-}} \StringTok{\textquotesingle{}https://www.imo{-}official.org/year\_statistics.aspx?year=1985\textquotesingle{}}

\CommentTok{\#using pipes to load the table into the data frame}
\FunctionTok{read\_html}\NormalTok{(url) }\SpecialCharTok{\%\textgreater{}\%} \FunctionTok{html\_table}\NormalTok{() }\SpecialCharTok{\%\textgreater{}\%} \FunctionTok{data.frame}\NormalTok{() }\OtherTok{{-}\textgreater{}}\NormalTok{ df}

\CommentTok{\#removing extra rows}
\NormalTok{df }\OtherTok{\textless{}{-}}\NormalTok{ df[}\SpecialCharTok{{-}}\FunctionTok{c}\NormalTok{(}\DecValTok{9}\SpecialCharTok{:}\DecValTok{20}\NormalTok{),]}

\CommentTok{\#adding a name to the first column, since it doesn\textquotesingle{}t have one in the webpage}
\FunctionTok{colnames}\NormalTok{(df)[}\DecValTok{1}\NormalTok{] }\OtherTok{=} \StringTok{"Problem Number"}

\CommentTok{\#adding the year to the table}
\NormalTok{df }\OtherTok{\textless{}{-}}\NormalTok{ df }\SpecialCharTok{\%\textgreater{}\%} \FunctionTok{mutate}\NormalTok{(}\AttributeTok{Problem\_year =} \DecValTok{1985}\NormalTok{)}

\CommentTok{\#our log super table, establishing log as the table where all the other tables will}
\CommentTok{\#get parsed}
\NormalTok{log }\OtherTok{=}\NormalTok{ df}
\NormalTok{log}
\end{Highlighting}
\end{Shaded}

\begin{verbatim}
##   Problem Number  P1 P2  P3 P4  P5 P6 Problem_year
## 1  Num( P# = 0 )  32 60 153 61 113 57         1985
## 2  Num( P# = 1 )  46 27  27 28  21 74         1985
## 3  Num( P# = 2 )  11  8   8 46  20 19         1985
## 4  Num( P# = 3 )   9  9   5 18   7 13         1985
## 5  Num( P# = 4 )   4  3   1 16   7  9         1985
## 6  Num( P# = 5 )   2  4   0  6   5 10         1985
## 7  Num( P# = 6 )   2  6   3  2   1  4         1985
## 8  Num( P# = 7 ) 103 92  12 32  35 23         1985
\end{verbatim}

\hypertarget{looping-through-the-years}{%
\subsection{Looping through the years}\label{looping-through-the-years}}

Now all that we have to do is to use a for-loop to repeat the process
and append every table of every year to the log super-table, this can be
achieved as follows:

\begin{Shaded}
\begin{Highlighting}[]
\CommentTok{\#creating a list of years to be scraped }
\NormalTok{Dates }\OtherTok{\textless{}{-}} \FunctionTok{c}\NormalTok{(}\DecValTok{1986}\SpecialCharTok{:}\DecValTok{2023}\NormalTok{)}

\CommentTok{\#start of for{-}loop}
\ControlFlowTok{for}\NormalTok{ (year }\ControlFlowTok{in}\NormalTok{ Dates)\{}
  
  \CommentTok{\#pasting the year as a string to the end of each URL to get the desired webpage}
\NormalTok{  url }\OtherTok{\textless{}{-}} \FunctionTok{paste}\NormalTok{(}\StringTok{\textquotesingle{}https://www.imo{-}official.org/year\_statistics.aspx?year=\textquotesingle{}}\NormalTok{,}\FunctionTok{toString}\NormalTok{(year),}\AttributeTok{sep =}\StringTok{""}\NormalTok{)}
  
  \CommentTok{\#using the df dummy variable to store the scraped raw data}
\NormalTok{  url }\SpecialCharTok{\%\textgreater{}\%}\NormalTok{ read\_html }\SpecialCharTok{\%\textgreater{}\%} \FunctionTok{html\_table}\NormalTok{() }\SpecialCharTok{\%\textgreater{}\%} \FunctionTok{data.frame}\NormalTok{() }\OtherTok{{-}\textgreater{}}\NormalTok{ df}
  
  \CommentTok{\#removing unwanted rows}
\NormalTok{  df }\OtherTok{\textless{}{-}}\NormalTok{ df[}\SpecialCharTok{{-}}\FunctionTok{c}\NormalTok{(}\DecValTok{9}\SpecialCharTok{:}\DecValTok{20}\NormalTok{),]}
  
  \CommentTok{\#filling the empty column name}
  \FunctionTok{colnames}\NormalTok{(df)[}\DecValTok{1}\NormalTok{] }\OtherTok{=} \StringTok{"Problem Number"}
  
  \CommentTok{\#adding the year as an extra column and to make later cleaning easier}
\NormalTok{  df }\OtherTok{\textless{}{-}}\NormalTok{ df }\SpecialCharTok{\%\textgreater{}\%} \FunctionTok{mutate}\NormalTok{(}\AttributeTok{Problem\_year =}\NormalTok{ year)}
  
  \CommentTok{\#adding the dummy variable df\textquotesingle{}s table to the log super{-}table, and preparing to}
  \CommentTok{\#repeat the loop}
\NormalTok{  log }\OtherTok{\textless{}{-}} \FunctionTok{bind\_rows}\NormalTok{(log, df)}

\NormalTok{\}}
\end{Highlighting}
\end{Shaded}

\hypertarget{saving-the-results-as-.csv}{%
\subsection{Saving the results as
.csv}\label{saving-the-results-as-.csv}}

Analysis and visualizations on
\href{https://public.tableau.com/views/InternationalMathematicalOlympiadDataCategoryandDifficulty/Dashboard1?:language=en-GB\&:sid=\&:display_count=n\&:origin=viz_share_link}{Tableau}

\begin{Shaded}
\begin{Highlighting}[]
\FunctionTok{write.csv}\NormalTok{(log, }\StringTok{"Name\_of\_your\_file.csv"}\NormalTok{)}
\end{Highlighting}
\end{Shaded}

\begin{Shaded}
\begin{Highlighting}[]
\NormalTok{htmltools}\SpecialCharTok{::}\FunctionTok{includeHTML}\NormalTok{(}\StringTok{"rrrr.html"}\NormalTok{)}
\end{Highlighting}
\end{Shaded}

\begin{verbatim}
## Warning: `includeHTML()` was provided a `path` that appears to be a complete HTML document.
## x Path: rrrr.html
## i Use `tags$iframe()` to include an HTML document. You can either ensure `path` is accessible in your app or document (see e.g. `shiny::addResourcePath()`) and pass the relative path to the `src` argument. Or you can read the contents of `path` and pass the contents to `srcdoc`.
\end{verbatim}

\end{document}
